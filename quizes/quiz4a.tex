\documentclass[10pt]{article}

\usepackage[margin=1in]{geometry}
\usepackage{amsmath}
\usepackage{amssymb}
\usepackage{amsthm}
\usepackage{mathtools}
\usepackage[shortlabels]{enumitem}

\usepackage{hyperref}
\hypersetup{
  colorlinks   = true, %Colours links instead of ugly boxes
  urlcolor     = black, %Colour for external hyperlinks
  linkcolor    = blue, %Colour of internal links
  citecolor    = blue  %Colour of citations
}

\usepackage{listings}
\lstset{language=Python} %,numbers=left}

%%%%%%%%%%%%%%%%%%%%%%%%%%%%%%%%%%%%%%%%%%%%%%%%%%%%%%%%%%%%%%%%%%%%%%%%%%%%%%%%

\theoremstyle{definition}
\newtheorem{problem}{Problem}
\newcommand{\E}{\mathbb E}
\newcommand{\R}{\mathbb R}
\DeclareMathOperator{\Var}{Var}
\DeclareMathOperator*{\argmin}{arg\,min}
\DeclareMathOperator*{\argmax}{arg\,max}

\newcommand{\trans}[1]{{#1}^{T}}
\newcommand{\loss}{\ell}
\newcommand{\w}{\mathbf w}
\newcommand{\mle}[1]{\hat{#1}_{\textit{mle}}}
\newcommand{\map}[1]{\hat{#1}_{\textit{map}}}
\newcommand{\normal}{\mathcal{N}}
\newcommand{\x}{\mathbf x}
\newcommand{\y}{\mathbf y}
\newcommand{\ltwo}[1]{\lVert {#1} \rVert}

%%%%%%%%%%%%%%%%%%%%%%%%%%%%%%%%%%%%%%%%%%%%%%%%%%%%%%%%%%%%%%%%%%%%%%%%%%%%%%%%

\begin{document}

\begin{center}
    {
\Large
In-class Quiz 4a (practice, do not turn in)
}

    \vspace{0.1in}
CSCI040, Computing for the Web

    \vspace{0.1in}
\end{center}

\vspace{0.25in}
\noindent
\textbf{Total Score:} ~~~~~~~~~~~~~~~/10

\vspace{0.5in}
\noindent
\textbf{Name:} (2pt)

\noindent
\rule{\textwidth}{0.1pt}
\vspace{0.25in}

\begin{problem}
    (2pt)
    The following code (circle one)
    
    \vspace{0.25in}
    \hspace{0.5in}terminates successfully
    \hspace{1in}runs forever
    \hspace{1in}generates an error
    \vspace{0.25in}

    \noindent
    If the code terminates successfully, what what is the output of the code?
    If the code runs forever or generates an error, explain why.
\end{problem}
\begin{lstlisting}
    d={}
    while len(d)<5:
        d[len(d)]='test'
    print('len(d)=',len(d))
\end{lstlisting}
\vspace{1.5in}

\begin{problem}
    (2pt)
    The following code (circle one)
    
    \vspace{0.25in}
    \hspace{0.5in}terminates successfully
    \hspace{1in}runs forever
    \hspace{1in}generates an error
    \vspace{0.25in}

    \noindent
    If the code terminates successfully, what what is the output of the code?
    If the code runs forever or generates an error, explain why.
\end{problem}
\begin{lstlisting}
     d={}
     d.append('1','a')
     d.append('2','b')
     d.append('2','c')
     print(d['1'])
\end{lstlisting}
\vspace{2in}
\newpage
     \noindent
     The following two problems assume that the following dictionary is defined.
     \begin{lstlisting}
     grades={
         'alice':{'hw1':99,'hw2':88},
         'bob':{'hw1':82,'hw2':91},
     }
     \end{lstlisting}
\begin{problem}
    (2pt)
    The following code (circle one)
    
    \vspace{0.25in}
    \hspace{0.5in}terminates successfully
    \hspace{1in}runs forever
    \hspace{1in}generates an error
    \vspace{0.25in}

    \noindent
    If the code terminates successfully, what what is the output of the code?
    If the code runs forever or generates an error, explain why.
\end{problem}
\begin{lstlisting}
    for k,v in grades.items():
        print(v['hw1'])
\end{lstlisting}
\vspace{2.5in}

\begin{problem}
    (2pt)
    The following code (circle one)
    
    \vspace{0.25in}
    \hspace{0.5in}terminates successfully
    \hspace{1in}runs forever
    \hspace{1in}generates an error
    \vspace{0.25in}

    \noindent
    If the code terminates successfully, what what is the output of the code?
    If the code runs forever or generates an error, explain why.
\end{problem}
\begin{lstlisting}
    grades['charlie']=[('hw1',88),('hw2',72)]
    for k in grades['charlie'].keys():
        print(grades['charlie'][k])
\end{lstlisting}
\vspace{1.5in}

\end{document}


