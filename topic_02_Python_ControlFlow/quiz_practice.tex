\documentclass[10pt]{article}

\usepackage[margin=1in]{geometry}
\usepackage{amsmath}
\usepackage{amssymb}
\usepackage{amsthm}
\usepackage{mathtools}
\usepackage[shortlabels]{enumitem}
\usepackage[normalem]{ulem}
\usepackage{courier}

\usepackage{hyperref}
\hypersetup{
  colorlinks   = true, %Colours links instead of ugly boxes
  urlcolor     = black, %Colour for external hyperlinks
  linkcolor    = blue, %Colour of internal links
  citecolor    = blue  %Colour of citations
}

\usepackage[T1]{fontenc}
\usepackage{listings}
\lstset{
    language={}
    ,basicstyle=\ttfamily
    %,numbers=left
    ,breaklines=true
    }

%%%%%%%%%%%%%%%%%%%%%%%%%%%%%%%%%%%%%%%%%%%%%%%%%%%%%%%%%%%%%%%%%%%%%%%%%%%%%%%%

\theoremstyle{definition}
\newtheorem{problem}{Problem}
\newcommand{\E}{\mathbb E}
\newcommand{\R}{\mathbb R}
\DeclareMathOperator{\Var}{Var}
\DeclareMathOperator*{\argmin}{arg\,min}
\DeclareMathOperator*{\argmax}{arg\,max}

\newcommand{\trans}[1]{{#1}^{T}}
\newcommand{\loss}{\ell}
\newcommand{\w}{\mathbf w}
\newcommand{\mle}[1]{\hat{#1}_{\textit{mle}}}
\newcommand{\map}[1]{\hat{#1}_{\textit{map}}}
\newcommand{\normal}{\mathcal{N}}
\newcommand{\x}{\mathbf x}
\newcommand{\y}{\mathbf y}
\newcommand{\ltwo}[1]{\lVert {#1} \rVert}

%%%%%%%%%%%%%%%%%%%%%%%%%%%%%%%%%%%%%%%%%%%%%%%%%%%%%%%%%%%%%%%%%%%%%%%%%%%%%%%%

\begin{document}
\begin{center}
    {
\Large
    Topic 02 Quiz Practice
}

    \vspace{0.1in}
    CSCI040: \sout{Computing for the Web} Introduction to Hacking

    \vspace{0.1in}
\end{center}

\noindent
\textbf{NOTE:}
\begin{enumerate}
\item
The quiz is designed to test your ability to read python code and evaluate it manually.
\item
There will be 4 problems
Each problem will follow the form of one of the problems below.
I will change the numbers/operators to ensure that you understand how to walk through evaluating python code manually.
\item
Study hard and do well on this quiz.
All future material in class assumes that you can complete problems similar to these perfectly!
\end{enumerate}
\vspace{0.1in}

\begin{problem}
    What is the output of the following code:
\end{problem}
\begin{lstlisting}
x = 2 // 3
y = x * 4
z = y + 5
print("z=", z)
\end{lstlisting}
\vspace{0.1in}

\begin{problem}
    What is the output of the following code:
\end{problem}
\begin{lstlisting}
x = 3 // 2
y = x % 4
z = y - 5
print("z=", z)
\end{lstlisting}
\vspace{0.1in}

%\begin{problem}
    %What is the output of the following code:
%\end{problem}
%\begin{lstlisting}
%x = 3 // 2 # 4
%y = x % 4
%z = y - 5 # 2 - 1
%print("z=", z)
%\end{lstlisting}
%\vspace{0.1in}

\begin{problem}
    What is the output of the following code:
\end{problem}
\begin{lstlisting}
print(2 * 2 ** 2 + 3 * 4 % 2)
\end{lstlisting}
\vspace{0.1in}

\begin{problem}
    What is the output of the following code:
\end{problem}
\begin{lstlisting}
print(2 ** 2 * 2 // 3 - 4 + 2)
\end{lstlisting}
\vspace{0.1in}

\begin{problem}
    What is the output of the following code:
\end{problem}
\begin{lstlisting}
x = 4%5
y = 4//5
if x == 1 or y == 1:
    result = 0
else:
    result = 1
print('result=', result)
\end{lstlisting}
\vspace{0.1in}

\newpage
\begin{problem}
    What is the output of the following code:
\end{problem}
\begin{lstlisting}
x = 4-5
y = 5-4
if x == 1 and y == 1:
    result = 0
else:
    result = 1
print('result=', result)
\end{lstlisting}
\vspace{0.1in}

\begin{problem}
    What is the output of the following code:
\end{problem}
\begin{lstlisting}
if '':
    result = 0
else:
    result = 1
print('result=', result)
\end{lstlisting}
\vspace{0.1in}

\begin{problem}
    What is the output of the following code:
\end{problem}
\begin{lstlisting}
if 1.0:
    result = 0
else:
    result = 1
print('result=', result)
\end{lstlisting}
\vspace{0.1in}



\begin{problem}
    What is the output of the following code:
\end{problem}
\begin{lstlisting}
i = 0
total = 0
while i < 5:
    total = total + i
    i += 1
print('total=', total)
\end{lstlisting}
\vspace{0.1in}


\begin{problem}
    What is the output of the following code:
\end{problem}
\begin{lstlisting}
i = 3
total = 0
while i < 8:
    total = total * i
    i += 1
print('total=', total)
\end{lstlisting}
\vspace{0.1in}


\begin{problem}
    What is the output of the following code:
\end{problem}
\begin{lstlisting}
i = 34567
total = 0
while i > 0:
    total += 1
    i //= 10
print('total=', total)
\end{lstlisting}
\vspace{0.1in}


\begin{problem}
    What is the output of the following code:
\end{problem}
\begin{lstlisting}
i = 123
total = 0
while i:
    total += 1
    i //= 10
print('total=', total)
\end{lstlisting}
\vspace{0.1in}


\begin{problem}
    What is the output of the following code:
\end{problem}
\begin{lstlisting}
total = 0
for i in range(5):
    total = total - 1
print('total=', total)
\end{lstlisting}
\vspace{0.1in}

\begin{problem}
    What is the output of the following code:
\end{problem}
\begin{lstlisting}
total = 0
for i in range(3, 5):
    total = total + 1
print('total=', total)
\end{lstlisting}
\vspace{0.1in}



\begin{problem}
    What is the output of the following code:
\end{problem}
\begin{lstlisting}
total = 42
for i in range(10, 15, 1):
    total %= i
print("total=", total)
\end{lstlisting}
\vspace{0.1in}

\begin{problem}
    What is the output of the following code:
\end{problem}
\begin{lstlisting}
total = 0
for i in range(10, 0, -2):
    total -= i
print("total=", total)
\end{lstlisting}
\vspace{0.1in}



\begin{problem}
    What is the output of the following code:
\end{problem}
\begin{lstlisting}
result = 1
for i in range(5):
    if i<3:
        result *= 1
    else:
        result *= (-1)
print('result=', result)
\end{lstlisting}
\vspace{0.1in}

\newpage
\begin{problem}
    What is the output of the following code:
\end{problem}
\begin{lstlisting}
result = 1
for i in range(3, 5):
    if i<3:
        result += 1
    else:
        result -= 1
print('result=', result)
\end{lstlisting}
\vspace{0.1in}



\begin{problem}
    What is the output of the following code:
\end{problem}
\begin{lstlisting}
result = 1
for i in range(5):
    if i <= 3:
        result += i
    else:
        result += 1
print('result=', result)
\end{lstlisting}
\vspace{0.1in}

\begin{problem}
    What is the output of the following code:
\end{problem}
\begin{lstlisting}
result = 1
for i in range(5):
    if i > 3:
        result += i
    else:
        result += 1
print('result=', result)
\end{lstlisting}
\vspace{0.1in}

\begin{problem}
    What is the output of the following code:
\end{problem}
\begin{lstlisting}
total = 0
for i in range(10, 20, 5):
    if i%2 == 1 or i<15:
        total += i
print("total=", total)
\end{lstlisting}
\vspace{0.1in}



\begin{problem}
    What is the output of the following code:
\end{problem}
\begin{lstlisting}
total = 0
for i in range(0, 10, 3):
    if i%2 == 0 or i<5:
        total += i
print("total=", total)
\end{lstlisting}
\vspace{0.1in}

\end{document}


