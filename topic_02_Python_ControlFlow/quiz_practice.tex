\documentclass[10pt]{article}

\usepackage[margin=1in]{geometry}
\usepackage{amsmath}
\usepackage{amssymb}
\usepackage{amsthm}
\usepackage{mathtools}
\usepackage[shortlabels]{enumitem}
\usepackage[normalem]{ulem}
\usepackage{courier}

\usepackage{hyperref}
\hypersetup{
  colorlinks   = true, %Colours links instead of ugly boxes
  urlcolor     = black, %Colour for external hyperlinks
  linkcolor    = blue, %Colour of internal links
  citecolor    = blue  %Colour of citations
}

\usepackage[T1]{fontenc}
\usepackage{listings}
\lstset{
    language=HTML
    ,basicstyle=\ttfamily
    %,numbers=left
    ,breaklines=true
    }

%%%%%%%%%%%%%%%%%%%%%%%%%%%%%%%%%%%%%%%%%%%%%%%%%%%%%%%%%%%%%%%%%%%%%%%%%%%%%%%%

\theoremstyle{definition}
\newtheorem{problem}{Problem}
\newcommand{\E}{\mathbb E}
\newcommand{\R}{\mathbb R}
\DeclareMathOperator{\Var}{Var}
\DeclareMathOperator*{\argmin}{arg\,min}
\DeclareMathOperator*{\argmax}{arg\,max}

\newcommand{\trans}[1]{{#1}^{T}}
\newcommand{\loss}{\ell}
\newcommand{\w}{\mathbf w}
\newcommand{\mle}[1]{\hat{#1}_{\textit{mle}}}
\newcommand{\map}[1]{\hat{#1}_{\textit{map}}}
\newcommand{\normal}{\mathcal{N}}
\newcommand{\x}{\mathbf x}
\newcommand{\y}{\mathbf y}
\newcommand{\ltwo}[1]{\lVert {#1} \rVert}

%%%%%%%%%%%%%%%%%%%%%%%%%%%%%%%%%%%%%%%%%%%%%%%%%%%%%%%%%%%%%%%%%%%%%%%%%%%%%%%%

\begin{document}
\begin{center}
    {
\Large
    Topic 02 Quiz Practice
}

    \vspace{0.1in}
    CSCI040: \sout{Computing for the Web} Introduction to Hacking

    \vspace{0.1in}
\end{center}

\noindent
\textbf{NOTE:}
\begin{enumerate}
\item
The quiz is designed to test your ability to read python code and evaluate it manually.
\item
There will be 5 problems
Each problem will follow the form of one of the problems below.
I will change the numbers/operators to ensure that you understand how to walk through evaluating python code manually.
\item
Recall that this quiz can be retaken once with no penalty.
There will be about 10 more quizzes in this class,
and of those 10, only 1 can be retaken.
\item
Study hard and do well on this quiz.
All future material in class assumes that you can complete problems similar to these perfectly!
\end{enumerate}
\vspace{0.15in}

\begin{problem}
    What is the output of the following code:
\end{problem}
\begin{lstlisting}
x = 2 // 3
y = x * 4
z = y + 5
print("z=", z)
\end{lstlisting}
\vspace{1.5in}

\begin{problem}
    What is the output of the following code:
\end{problem}
\begin{lstlisting}
x = 3 // 2
y = x % 4
z = y - 5
print("z=", z)
\end{lstlisting}
\vspace{1.5in}

%\begin{problem}
    %What is the output of the following code:
%\end{problem}
%\begin{lstlisting}
%x = 3 // 2 # 4
%y = x % 4
%z = y - 5 # 2 - 1
%print("z=", z)
%\end{lstlisting}
%\vspace{1.5in}

\begin{problem}
    What is the output of the following code:
\end{problem}
\begin{lstlisting}
print(2 * 2 ** 2 + 3 * 4 % 2)
\end{lstlisting}
\vspace{1.5in}

\begin{problem}
    What is the output of the following code:
\end{problem}
\begin{lstlisting}
print(2 ** 2 * 2 // 3 - 4 + 2)
\end{lstlisting}
\vspace{1.5in}

\begin{problem}
    What is the output of the following code:
\end{problem}
\begin{lstlisting}
x = 4%5
y = 4//5
if x == 1 or y == 1:
    result = 0
else:
    result = 1
print('result=', result)
\end{lstlisting}
\vspace{1.8in}

\begin{problem}
    What is the output of the following code:
\end{problem}
\begin{lstlisting}
x = 4-5
y = 5-4
if x == 1 and y == 1:
    result = 0
else:
    result = 1
print('result=', result)
\end{lstlisting}
\vspace{1.8in}

\begin{problem}
    What is the output of the following code:
\end{problem}
\begin{lstlisting}
if '':
    result = 0
else:
    result = 1
print('result=', result)
\end{lstlisting}
\vspace{1.8in}

\begin{problem}
    What is the output of the following code:
\end{problem}
\begin{lstlisting}
if 1.0:
    result = 0
else:
    result = 1
print('result=', result)
\end{lstlisting}
\vspace{1.8in}



\begin{problem}
    What is the output of the following code:
\end{problem}
\begin{lstlisting}
i = 0
total = 0
while i < 5:
    total = total + i
    i += 1
print('total=', total)
\end{lstlisting}
\vspace{1.8in}


\begin{problem}
    What is the output of the following code:
\end{problem}
\begin{lstlisting}
i = 3
total = 0
while i < 8:
    total = total * i
    i += 1
print('total=', total)
\end{lstlisting}
\vspace{1.8in}


\begin{problem}
    What is the output of the following code:
\end{problem}
\begin{lstlisting}
i = 34567
total = 0
while i > 0:
    total += 1
    i //= 10
print('total=', total)
\end{lstlisting}
\vspace{1.8in}


\begin{problem}
    What is the output of the following code:
\end{problem}
\begin{lstlisting}
i = 123
total = 0
while i:
    total += 1
    i //= 10
print('total=', total)
\end{lstlisting}
\vspace{1.8in}


\begin{problem}
    What is the output of the following code:
\end{problem}
\begin{lstlisting}
total = 0
for i in range(5):
    total = total - 1
print('total=', total)
\end{lstlisting}
\vspace{2in}

\begin{problem}
    What is the output of the following code:
\end{problem}
\begin{lstlisting}
total = 0
for i in range(3, 5):
    total = total + 1
print('total=', total)
\end{lstlisting}
\vspace{2in}



\begin{problem}
    What is the output of the following code:
\end{problem}
\begin{lstlisting}
total = 42
for i in range(10, 15, 1):
    total %= i
print("total=", total)
\end{lstlisting}
\vspace{2in}

\begin{problem}
    What is the output of the following code:
\end{problem}
\begin{lstlisting}
total = 0
for i in range(10, 0, -2):
    total -= i
print("total=", total)
\end{lstlisting}
\vspace{2in}



\begin{problem}
    What is the output of the following code:
\end{problem}
\begin{lstlisting}
result = 1
for i in range(5):
    if i<3:
        result *= 1
    else:
        result *= (-1)
print('result=', result)
\end{lstlisting}
\vspace{2in}

\begin{problem}
    What is the output of the following code:
\end{problem}
\begin{lstlisting}
result = 1
for i in range(3, 5):
    if i<3:
        result += 1
    else:
        result -= 1
print('result=', result)
\end{lstlisting}
\vspace{2in}



\begin{problem}
    What is the output of the following code:
\end{problem}
\begin{lstlisting}
result = 1
for i in range(5):
    if i <= 3:
        result += i
    else:
        result += 1
print('result=', result)
\end{lstlisting}
\vspace{2in}

\begin{problem}
    What is the output of the following code:
\end{problem}
\begin{lstlisting}
result = 1
for i in range(5):
    if i > 3:
        result += i
    else:
        result += 1
print('result=', result)
\end{lstlisting}
\vspace{2in}

\begin{problem}
    What is the output of the following code:
\end{problem}
\begin{lstlisting}
total = 0
for i in range(10, 20, 5):
    if i%2 == 1 or i<15:
        total += i
print("total=", total)
\end{lstlisting}
\vspace{2in}



\begin{problem}
    What is the output of the following code:
\end{problem}
\begin{lstlisting}
total = 0
for i in range(0, 10, 3):
    if i%2 == 0 or i<5:
        total += i
print("total=", total)
\end{lstlisting}
\vspace{2in}


\begin{problem}
    What is the output of the following code:
\end{problem}
\begin{lstlisting}
x = 5
def foo():
    print('hello')
    return 1
x += foo()
print("x=", x)
\end{lstlisting}
\vspace{2in}


\begin{problem}
    What is the output of the following code:
\end{problem}
\begin{lstlisting}
x = 5
def foo():
    'goodbye'
    print('hello')
    return 1
x += foo()
print("x=", x)
\end{lstlisting}
\vspace{2in}


\begin{problem}
    What is the output of the following code:
\end{problem}
\begin{lstlisting}
x = 5
def foo():
    print('hello')
    return 1
\end{lstlisting}
\vspace{2in}


\begin{problem}
    What is the output of the following code:
\end{problem}
\begin{lstlisting}
x = 5
def foo():
    print('hello')
    return 1
x = foo
\end{lstlisting}
\vspace{2in}


\begin{problem}
    What is the output of the following code:
\end{problem}
\begin{lstlisting}
x = 5
def foo(y):
    print('hello')
    return y*2
x += foo(3)
print("x=", x)
\end{lstlisting}
\vspace{2in}


\begin{problem}
    What is the output of the following code:
\end{problem}
\begin{lstlisting}
x = 5
def foo(y):
    print('hello')
    return y*2
x += foo(3)
x += foo(2)
x += foo(1)
print("x=", x)
\end{lstlisting}
\vspace{2in}


\begin{problem}
    What is the output of the following code:
\end{problem}
\begin{lstlisting}
x = 5
def foo(y, z):
    print('hello')
    return y*z
x += foo(3, 4)
print("x=", x)
\end{lstlisting}
\vspace{2in}


\begin{problem}
    What is the output of the following code:
\end{problem}
\begin{lstlisting}
x = 5
def foo(y, z):
    print('z=', z)
    return y
x += foo(3, 4)
x += foo(4, 3)
print("x=", x)
\end{lstlisting}
\vspace{2in}


\begin{problem}
    What is the output of the following code:
\end{problem}
\begin{lstlisting}
x = 5
def foo(y, z):
    print('z=', z)
    return y
x += foo(y=3, z=4)
x += foo(z=4, y=3)
print("x=", x)
\end{lstlisting}
\vspace{2in}


\begin{problem}
    What is the output of the following code:
\end{problem}
\begin{lstlisting}
x = 5
def foo(x):
    x += 1
    return x
x += foo(4)
x += foo(5)
print("x=", x)
\end{lstlisting}
\vspace{2in}


\begin{problem}
    What is the output of the following code:
\end{problem}
\begin{lstlisting}
x = 5
def foo(x):
    x += 1
    return x
x += foo(9 + 19 // 10) + 3
print("x=", x)
\end{lstlisting}
\vspace{2in}


\begin{problem}
    What is the output of the following code:
\end{problem}
\begin{lstlisting}
x = 5
def foo(x):
    x += 1
    return x
x += foo(9 + foo(19) // 10) + 3
print("x=", x)
\end{lstlisting}
\vspace{2in}



\newpage
\begin{problem}
    What is the output of the following code:
\end{problem}
\begin{lstlisting}
x = 5
def foo(x):
    if x % 2:
        return 1
    else:
        return 2
x += foo(4)
x += foo(5)
x += foo(6)
print("x=", x)
\end{lstlisting}
\vspace{2in}


\begin{problem}
    What is the output of the following code:
\end{problem}
\begin{lstlisting}
x = 5
def foo(x):
    if x % 2:
        return 1
    x -= 1
    return x
x += foo(4)
x += foo(5)
x += foo(6)
print("x=", x)
\end{lstlisting}
\vspace{2in}


\newpage
\begin{problem}
    What is the output of the following code:
\end{problem}
\begin{lstlisting}
x = 5
def foo(x):
    if x:
        return 1
    else:
        return -1
x += foo(4)
x += foo(0)
x += foo('')
x += foo('test')
x += foo("")
x += foo(False)
print("x=", x)
\end{lstlisting}
\vspace{2in}


\begin{problem}
    What is the output of the following code:
\end{problem}
\begin{lstlisting}
x = 5
def foo(x):
    res = 0
    for i in range(x):
        res += 1
    return res
x += foo(1)
x += foo(2)
x += foo(3)
print("x=", x)
\end{lstlisting}
\vspace{2in}


\newpage
\begin{problem}
    What is the output of the following code:
\end{problem}
\begin{lstlisting}
x = 5
def foo(x):
    res = 0
    for i in range(x):
        if i:
            res += 1
    return res
x += foo(1)
x += foo(2)
x += foo(3)
print("x=", x)
\end{lstlisting}
\vspace{2in}


\begin{problem}
    What is the output of the following code:
\end{problem}
\begin{lstlisting}
x = 5
def foo(x):
    res = 0
    for i in range(x):
        res += 1
    return res
for i in range(3):
    x += foo(i)
print("x=", x)
\end{lstlisting}
\vspace{2in}


\newpage
\begin{problem}
    What is the output of the following code:
\end{problem}
\begin{lstlisting}
x = 5
def foo(x):
    res = 0
    for i in range(x):
        res += 1
    return res
for i in range(3):
    x += foo(i)
print("x=", x)
\end{lstlisting}
\vspace{2in}


\begin{problem}
    What is the output of the following code:
\end{problem}
\begin{lstlisting}
def foo(x):
    total = 0
    while x > 0:
        total += 1
        x //= 10
    return total
x = foo(100)
x += foo(1234567)
x += foo(3)
print("x=", x)
\end{lstlisting}
\vspace{2in}

\newpage
\begin{problem}
    What is the output of the following code:
\end{problem}
\begin{lstlisting}
def foo(x):
    total = 0
    while x > 0:
        total = total + x % 10
        x //= 10
    return total
x = foo(100)
x += foo(1234567)
x += foo(3)
print("x=", x)
\end{lstlisting}
\vspace{2in}

\end{document}


