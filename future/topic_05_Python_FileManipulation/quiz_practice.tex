\documentclass[10pt]{article}

\usepackage[margin=1in]{geometry}
\usepackage{amsmath}
\usepackage{amssymb}
\usepackage{amsthm}
\usepackage{mathtools}
\usepackage[shortlabels]{enumitem}
\usepackage[normalem]{ulem}

\usepackage{hyperref}
\hypersetup{
  colorlinks   = true, %Colours links instead of ugly boxes
  urlcolor     = black, %Colour for external hyperlinks
  linkcolor    = blue, %Colour of internal links
  citecolor    = blue  %Colour of citations
}

\usepackage{listings}
\lstset{language=Python} %,numbers=left}

%%%%%%%%%%%%%%%%%%%%%%%%%%%%%%%%%%%%%%%%%%%%%%%%%%%%%%%%%%%%%%%%%%%%%%%%%%%%%%%%

\theoremstyle{definition}
\newtheorem{problem}{Problem}
\newcommand{\E}{\mathbb E}
\newcommand{\R}{\mathbb R}
\DeclareMathOperator{\Var}{Var}
\DeclareMathOperator*{\argmin}{arg\,min}
\DeclareMathOperator*{\argmax}{arg\,max}

\newcommand{\trans}[1]{{#1}^{T}}
\newcommand{\loss}{\ell}
\newcommand{\w}{\mathbf w}
\newcommand{\mle}[1]{\hat{#1}_{\textit{mle}}}
\newcommand{\map}[1]{\hat{#1}_{\textit{map}}}
\newcommand{\normal}{\mathcal{N}}
\newcommand{\x}{\mathbf x}
\newcommand{\y}{\mathbf y}
\newcommand{\ltwo}[1]{\lVert {#1} \rVert}

%%%%%%%%%%%%%%%%%%%%%%%%%%%%%%%%%%%%%%%%%%%%%%%%%%%%%%%%%%%%%%%%%%%%%%%%%%%%%%%%

\begin{document}

\begin{center}
    {
\Large
    Week 06 Practice Quiz
}

    \vspace{0.1in}
    CSCI040: \sout{Computing for the Web} Introduction to Hacking

    \vspace{0.1in}
\end{center}


\vspace{0.15in}
\noindent
\textbf{Total Score:} ~~~~~~~~~~~~~~~/10

\vspace{0.5in}
\noindent
\textbf{Printed Name:}

\noindent
\rule{\textwidth}{0.1pt}
\vspace{0.25in}

\noindent
\textbf{Collaboration Policy:}
\begin{enumerate}
    \item You MAY use any printed or handwritten notes.
    \item You MAY NOT use a computer or any other electronic device.
    \item You MAY NOT discuss this quiz with another human being who has not completed the quiz.
        This includes:
        \begin{enumerate}
            \item collaborating during the quiz, and
            \item telling a student in a different section the quiz was easy/hard.
        \end{enumerate}
\end{enumerate}

\noindent
\textbf{Special Notes:}
\begin{enumerate}
    \item Each problem worth 2 points.
    \item Your grade on this quiz can replace lowest grade on a previous python quiz.
\end{enumerate}

\vspace{0.15in}

\begin{problem}
    What is the output of the following code:
\end{problem}
\begin{lstlisting}
xss = [[1, 3, 5], [2, 4], [0, 1, 2, 3, 4, 5]]
total = 0
for i in range(2):
    for j in range(1, 2):
        total += xss[i][j]
print('total=', total)
\end{lstlisting}
\vspace{1in}

\begin{problem}
    What is the output of the following code:
\end{problem}
\begin{lstlisting}
s = '12345'
xs = ['1', '2', '3', '4', '5']
b = s[-1:] == xs[-1:]
print('b=', b)
\end{lstlisting}
\vspace{1.5in}

\newpage
\begin{problem}
    What is the output of the following code:
\end{problem}
\begin{lstlisting}
print(3 * 1 ** 6 + 2 * 4 % 10)
\end{lstlisting}
\vspace{1.5in}


\begin{problem}
    What is the output of the following code:
\end{problem}
\begin{lstlisting}
i = 0
total = 0
xs = [0, 1, 2, 3]
while xs[i]:
    total = total + i
    i += 1
print('total=', total)
\end{lstlisting}
\vspace{1.5in}


\begin{problem}
    What is the output of the following code:
\end{problem}
\begin{lstlisting}
total = 0
for i in range(0x01, 0x10, 0x4):
    total -= i
print("total=", total)
\end{lstlisting}
\vspace{1.5in}


\end{document}
