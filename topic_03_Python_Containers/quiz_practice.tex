\documentclass[10pt]{article}

\usepackage[margin=1in]{geometry}
\usepackage{amsmath}
\usepackage{amssymb}
\usepackage{amsthm}
\usepackage{mathtools}
\usepackage[shortlabels]{enumitem}
\usepackage[normalem]{ulem}

\usepackage{hyperref}
\hypersetup{
  colorlinks   = true, %Colours links instead of ugly boxes
  urlcolor     = black, %Colour for external hyperlinks
  linkcolor    = blue, %Colour of internal links
  citecolor    = blue  %Colour of citations
}

\usepackage{listings}
\lstset{language=Python} %,numbers=left}

%%%%%%%%%%%%%%%%%%%%%%%%%%%%%%%%%%%%%%%%%%%%%%%%%%%%%%%%%%%%%%%%%%%%%%%%%%%%%%%%

\theoremstyle{definition}
\newtheorem{problem}{Problem}
\newtheorem{note}{Note}
\newcommand{\E}{\mathbb E}
\newcommand{\R}{\mathbb R}
\DeclareMathOperator{\Var}{Var}
\DeclareMathOperator*{\argmin}{arg\,min}
\DeclareMathOperator*{\argmax}{arg\,max}

\newcommand{\trans}[1]{{#1}^{T}}
\newcommand{\loss}{\ell}
\newcommand{\w}{\mathbf w}
\newcommand{\mle}[1]{\hat{#1}_{\textit{mle}}}
\newcommand{\map}[1]{\hat{#1}_{\textit{map}}}
\newcommand{\normal}{\mathcal{N}}
\newcommand{\x}{\mathbf x}
\newcommand{\y}{\mathbf y}
\newcommand{\ltwo}[1]{\lVert {#1} \rVert}

%%%%%%%%%%%%%%%%%%%%%%%%%%%%%%%%%%%%%%%%%%%%%%%%%%%%%%%%%%%%%%%%%%%%%%%%%%%%%%%%

\begin{document}

\begin{center}
    {
\Large
Containers Quiz Practice
}

    \vspace{0.1in}
    CSCI040: \sout{Computing for the Web} Introduction to Hacking

    \vspace{0.1in}
\end{center}


\vspace{0.15in}

\begin{problem}
    What is the output of the following code:
\end{problem}
\begin{lstlisting}
xs = [1, 3, 5, 7, 9, 11, 13, 15, 17, 19, 21]
total = 0
total += xs[0]
total += xs[1]
total += xs[-3]
print('total=', total)
\end{lstlisting}
\vspace{0.15in}

\begin{problem}
    What is the output of the following code:
\end{problem}
\begin{lstlisting}
xs = [1, 3, 5, 7, 9, 11, 13, 15, 17, 19, 21]
total = 0
total += xs[3]
total += xs[-1]
total += xs[0]
print('total=', total)
\end{lstlisting}
\vspace{0.15in}

\begin{problem}
    What is the output of the following code:
\end{problem}
\begin{lstlisting}
xs = [1, 3, 5, 7, 9, 11, 13, 15, 17, 19, 21]
total = 0
total += xs[5]
total += xs[-5]
total += xs[-4]
print('total=', total)
\end{lstlisting}
\vspace{0.15in}

\begin{problem}
    What is the output of the following code:
\end{problem}
\begin{lstlisting}
xs = [1, 3, 5, 7, 9, 11, 13, 15, 17, 19, 21]
ys = xs[3:5]
total = 0
total += ys[0]
total += ys[1]
total += ys[-1]
print('total=', total)
\end{lstlisting}
\vspace{0.15in}

\begin{problem}
    What is the output of the following code:
\end{problem}
\begin{lstlisting}
xs = [1, 3, 5, 7, 9, 11, 13, 15, 17, 19, 21]
ys = xs[-5:-3]
total = 0
total += ys[0]
total += ys[1]
total += ys[-1]
print('total=', total)
\end{lstlisting}
\vspace{0.15in}

\begin{problem}
    What is the output of the following code:
\end{problem}
\begin{lstlisting}
xs = [1, 3, 5, 7, 9, 11, 13, 15, 17, 19, 21]
ys = xs[-5:-3]
total = len(ys)
print('total=', total)
\end{lstlisting}
\vspace{0.15in}

\begin{problem}
    What is the output of the following code:
\end{problem}
\begin{lstlisting}
xs = [1, 3, 5, 7, 9, 11, 13, 15, 17, 19, 21]
ys = xs[-5:-3]
total = sum(ys)
print('total=', total)
\end{lstlisting}
\vspace{0.15in}

\begin{problem}
    What is the output of the following code:
\end{problem}
\begin{lstlisting}
xs = [1, 3, 5, 7, 9, 11, 13, 15, 17, 19, 21]
ys = xs[-3:-5]
total = sum(ys)
print('total=', total)
\end{lstlisting}
\vspace{0.15in}

\begin{problem}
    What is the output of the following code:
\end{problem}
\begin{lstlisting}
xs = [1, 3, 5, 7, 9, 11, 13, 15, 17, 19, 21]
ys = xs[-3:-5:-1]
total = sum(ys)
print('total=', total)
\end{lstlisting}
\vspace{0.15in}

\begin{problem}
    What is the output of the following code:
\end{problem}
\begin{lstlisting}
xs = [1, 3, 5, 7, 9, 11, 13, 15, 17, 19, 21]
ys = xs[:3]
total = sum(ys)
print('total=', total)
\end{lstlisting}
\vspace{0.15in}

\begin{problem}
    What is the output of the following code:
\end{problem}
\begin{lstlisting}
xs = [1, 3, 5, 7, 9, 11, 13, 15, 17, 19, 21]
ys = xs[:3]
total = min(ys)
print('total=', total)
\end{lstlisting}
\vspace{0.15in}

\begin{problem}
    What is the output of the following code:
\end{problem}
\begin{lstlisting}
xs = [1, 3, 5, 7, 9, 11, 13, 15, 17, 19, 21]
ys = xs[:3]
total = max(ys)
print('total=', total)
\end{lstlisting}
\vspace{0.15in}

\begin{problem}
    What is the output of the following code:
\end{problem}
\begin{lstlisting}
xs = [1, 3, 5, 7, 9, 11, 13, 15, 17, 19, 21]
ys = xs[7:]
total = sum(ys)
print('total=', total)
\end{lstlisting}
\vspace{0.15in}

\begin{problem}
    What is the output of the following code:
\end{problem}
\begin{lstlisting}
xs = [1, 3, 5, 7, 9, 11, 13, 15, 17, 19, 21]
ys = xs[-3:]
total = sum(ys)
print('total=', total)
\end{lstlisting}
\vspace{0.15in}

\begin{problem}
    What is the output of the following code:
\end{problem}
\begin{lstlisting}
xs = [1, 3, 5, 7, 9, 11, 13, 15, 17, 19, 21]
total = 0
for i in range(3):
    total += xs[i]
print('total=', total)
\end{lstlisting}
\vspace{0.15in}

\begin{problem}
    What is the output of the following code:
\end{problem}
\begin{lstlisting}
xs = [1, 3, 5, 7, 9, 11, 13, 15, 17, 19, 21]
total = 0
for i in range(3, 8, 2):
    total += xs[i]
print('total=', total)
\end{lstlisting}
\vspace{0.15in}

\begin{problem}
    What is the output of the following code:
\end{problem}
\begin{lstlisting}
xs = [1, 3, 5, 7, 9, 11, 13, 15, 17, 19, 21]
total = 0
for i in range(-3, -6, -1):
    total += xs[i]
print('total=', total)
\end{lstlisting}
\vspace{0.15in}

\begin{problem}
    What is the output of the following code:
\end{problem}
\begin{lstlisting}
xs = [1, 3, 5]
total = 0
for x in xs:
    total += x
print('total=', total)
\end{lstlisting}
\vspace{0.15in}

\newpage
\begin{problem}
    What is the output of the following code:
\end{problem}
\begin{lstlisting}
xs = [1, 3, 5]
total = 0
for x in xs:
    for i in range(3):
        total += x*i
print('total=', total)
\end{lstlisting}
\vspace{0.15in}


\begin{problem}
    What is the output of the following code:
\end{problem}
\begin{lstlisting}
xs = [1, 3, 5]
ys = [2, 4, 6]
total = 0
for x in xs:
    total += x
    for y in ys:
        total -= x*y
print('total=', total)
\end{lstlisting}
\vspace{0.15in}


\begin{problem}
    What is the output of the following code:
\end{problem}
\begin{lstlisting}
xss = [[1, 3, 5], [2, 4], [0, 1, 2, 3, 4, 5]]
total = 0
total += xss[0][0]
total += xss[1][1]
total += xss[2][2]
print('total=', total)
\end{lstlisting}
\vspace{0.15in}

\begin{problem}
    What is the output of the following code:
\end{problem}
\begin{lstlisting}
xss = [[1, 3, 5], [2, 4], [0, 1, 2, 3, 4, 5]]
total = 0
total += xss[1][0]
total += xss[0][1]
total += xss[0][2]
print('total=', total)
\end{lstlisting}
\vspace{0.15in}

\begin{problem}
    What is the output of the following code:
\end{problem}
\begin{lstlisting}
xss = [[1, 3, 5], [2, 4], [0, 1, 2, 3, 4, 5]]
total = 0
total += xss[-1][0]
total += xss[0][-1]
total += xss[-2][2]
print('total=', total)
\end{lstlisting}
\vspace{0.15in}

\newpage
\begin{problem}
    What is the output of the following code:
\end{problem}
\begin{lstlisting}
xss = [[1, 3, 5], [2, 4], [0, 1, 2, 3, 4, 5]]
total = 0
for xs in xss:
    total += xs[0]
    for x in xs:
        total += x
print('total=', total)
\end{lstlisting}
\vspace{0.15in}

\begin{problem}
    What is the output of the following code:
\end{problem}
\begin{lstlisting}
xss = [[1, 3, 5], [2, 4], [0, 1, 2, 3, 4, 5]]
total = 0
for xs in xss:
    total += xs[0]
    for x in xs:
        total %= x
print('total=', total)
\end{lstlisting}
\vspace{0.15in}


\begin{problem}
    What is the output of the following code:
\end{problem}
\begin{lstlisting}
xss = [[1, 3, 5], [2, 4], [0, 1, 2, 3, 4, 5]]
total = 0
for i in range(2):
    for j in range(len(xss[i])):
        total += xss[i][-j]
print('total=', total)
\end{lstlisting}
\vspace{0.15in}

\newpage
\begin{problem}
    What is the output of the following code:
\end{problem}
\begin{lstlisting}
x = 10
def foo(x):
    if x - 5:
        return 1
    else:
        x += 1
    return x
x += foo(4)
x += foo(5)
x += foo(6)
print("x=", x)
\end{lstlisting}
\vspace{0.15in}

\begin{problem}
    What is the output of the following code:
\end{problem}
\begin{lstlisting}
x = 10 
def foo(x):
    x += 2
    return x
x += foo(9 + 39 // 10) * 3
x += foo(9 + 19 // 10) * 2
print("x=", x)
\end{lstlisting}
\vspace{0.15in}

\begin{problem}
    What is the output of the following code:
\end{problem}
\begin{lstlisting}
x = 10
def foo(x):
    for i in range(1, 4):
        return x * i
x += foo(i)
print("x=", x)
\end{lstlisting}
\vspace{0.15in}

\begin{problem}
    What is the output of the following code:
\end{problem}
\begin{lstlisting}
x = 10
def foo(x):
    if x % 2:
        return 1
    x -= 1
    return x
x += foo(4)
x += foo(5)
x += foo(6)
print("x=", x)
\end{lstlisting}
\vspace{0.15in}

\newpage
\begin{problem}
    What is the output of the following code:
\end{problem}
\begin{lstlisting}
x = 10 
def foo(x):
    x += 1
    return x
x += foo(9 + 39 // 10) * 2
print("x=", x)
\end{lstlisting}
\vspace{0.15in}

\begin{problem}
    What is the output of the following code:
\end{problem}
\begin{lstlisting}
x = 10
def foo(x):
    return x * 2
for i in range(3):
    x += foo(i)
print("x=", x)
\end{lstlisting}
\vspace{0.15in}
\begin{problem}
    What is the output of the following code:
\end{problem}
\begin{lstlisting}
x = 10
def foo(x):
    if x - 5:
        return 1
    else:
        x -= 1
    return x
x += foo(4)
x += foo(5)
x += foo(6)
print("x=", x)
\end{lstlisting}
\vspace{0.15in}

\begin{problem}
    What is the output of the following code:
\end{problem}
\begin{lstlisting}
x = 10 
def foo(x):
    x += 2
    return x
x += foo(9 + 39 // 10) * 3
x += foo(9 + 39 // 10) * 2
print("x=", x)
\end{lstlisting}
\vspace{0.15in}

\begin{problem}
    What is the output of the following code:
\end{problem}
\begin{lstlisting}
x = 10
def foo(x):
    for i in range(3):
        return x * i
x += foo(i)
print("x=", x)
\end{lstlisting}
\vspace{0.15in}

\begin{problem}
    What is the output of the following code:
\end{problem}
\begin{lstlisting}
def foo(x):
    total = 0
    while x > 0:
        total += 1
        x //= 10
    return total
x = foo(100)
x += foo(1234567)
x += foo(3)
print("x=", x)
\end{lstlisting}
\vspace{0.15in}

\begin{problem}
    What is the output of the following code:
\end{problem}
\begin{lstlisting}
def foo(x):
    total = 0
    while x > 0:
        total = total + x % 10
        x //= 10
    return total
x = foo(100)
x += foo(1234567)
x += foo(3)
print("x=", x)
\end{lstlisting}
\vspace{0.15in}

\end{document}
